\documentclass{trlnotes}
\usepackage{trmath}
\addcompatiblelayout{thesis}
\setlayout[size=thesis12]{thesis}

\usepackage{aas_macros}
\usepackage[
  backend=biber,
  natbib=true,
  style=authoryear-comp,
  sorting=none,
]{biblatex}
\addbibresource{thesisrefs.bib}
\begin{document}

Одна из наиболее ярких особенностей галактик, наблюдаемых <<плашмя>>~"--- это бары:
вытянутые образования, выделяющиеся на фоне остальной галактики.
Они часто возникают в расчётах, поскольку звёздные диски в $N$-body моделях 
как правило неустойчивы относительно бароподобных возмущений.
\note{в этом месте мне почему-то стало казаться что <<неустойчивости относительно формирования бара>>~"--- калька с английского,
но всегда можно вернуть обратно} Для подавления формирования бара необходимы специальные условия, хороший обзор которых приведён 
в \citet{sellwood2019}.

Многие интересные наблюдательные особенности галактик так или иначе связаны с барами. 
Так, например, при взгляде с ребра на диск галактики, его центральная части может иметь 
<<ящикообразную>> или даже <<арахисообразную>> форму~"--- содержать так называемый B/PS балдж. 
Многочисленные теоретические исследования и наблюдения установили надёжную связь этих структур с барами, <<изгибающимися>>\note{
  рецензенту не нравится такой перевод, но я не знаю как лучше
} в вертикальном направлении в ходе эволюции галактики.
Не менее интересны морфологические детали баров, когда диск галактики виден <<плашмя>>. Бары в таком случае можно разделить на две 
группы: с ящикообразными изофотами~"--- арахисообразные бары и с округлыми~"--- бары с барлинзами. 
Изучение природы этих морфологических различий является предметом данной работы.
Для того, чтобы понять чем обусловлены различия в форме баров, необходимо детально исследовать их орбитальный состав
и сопоставить его с параметрами родительской галактики.

В работе был проведён анализ орбитального состава диска для 4-x $N$-body моделей с барами,
отличающихся массой и характерным радиусом классического балджа. Основное внимание было уделено морфологии орбит в плоскости диска. 
В полном согласии результатами, описанными в литературе, было найдено, что по мере увеличения центральной концентрации балджа 
изменяется морфология бара <<плашмя>>~"--- происходит плавный переход от арахисообразного бара к бару с барлинзой.

Для исследования орбитальной структуры изучаемых моделей был использован метод анализа частот, впервые описанный в \citet{binney1982}.
Поскольку орбиты, определяющие долгоживущую структуру в звёздном диске, должны быть квазипериодическими, их спектры должны состоять из
дискретного набора пиков. Для всех 4 млн. частиц звёздного диска в каждой из моделей были рассчитаны доминирующие частоты
$ω_x, ω_y, ω_z, ω_R$ соответствующие наиболее высоким пикам на периодограммах временных рядов декартовых координат и цилиндрического радиуса. 
С помощью полученных значений и их отношений в исследуемых моделях был выделен весь бар. 
Далее, была разработана схема классификации орбитальных семейств в нём.
Сравнение населённости различных семейств в различных моделях показало, что арахисообразная морфология бара создаётся, в 
основном, ящикообразными орбитами, находящимися на внутреннем резонансе Линдблада. Это подтверждает результаты предыдущих работ, основанных как на анализе большого числа орбит в <<замороженных>> потенциалах, так и на анализе скромного числа выборочных орбит в $N$-body потенциалах. 
Был сделан вывод, что основная роль в формировании округлых очертаний бара с барлинзой принадлежит семействам, не поддерживающим бар. 
Разработанная классификация позволила выделить компактное орбитальное семейство, находящееся в центральных областях бара. 
На движение звёзд, входящих в него, наибольшее влияние оказывает потенциал балджа. 
Именно он создаёт условия для рассеяния орбит в центральных областях и препятствует захвату быстропрецессирующих орбит в бар.
На мой взгляд, это семейство является ключевым компонентом барлинзы, без которого невозможно получить характерную для неё округлую форму изофот.
Несмотря на его значимость, оно ещё ни разу не выделялось в подобных исследованиях.
Другой интересный результат, полученный в работе~"--- вывод, что точная форма изофот бара с барлинзой определяется тонким балансом между
населённостью различных семейств. В частности, при уменьшении концентрации балджа в модели на первый план выходит более протяжённое, 
не поддерживающие бар семейство, придающее барлинзе скорее ящикообразные, чем округлые изофоты. 

Исследование кинематики и эволюции выделенных семейств является предметом для дальнейших исследований, продолжающих эту работу.



\newpage
В дипломной работе рецензентом было выявлено несколько недочётов.
В частности, я согласен с указанной неточностью в выкладке на странице 12, где вместо цилиндрического радиуса 
дифференцирование ведётся по радиальной координате. Поскольку рассматриваются частицы только в плоскости диска
галактики, при подстановке $r=R$ правильная формула отличается от приведённой в тексте дипломной работы только
обозначениями, и эта ошибка на дальнейшее изложение не влияет.

Рецензентом также было отмечено избыточное использование изложения от первого лица, не принятое в научных работах.
Я согласен с этим замечанием и постараюсь в дальнейшем избегать подобного стиля. 
Насколько мне известно, дипломные работы и диссертации зачастую выступают исключением из этого правила.
Поскольку дипломная работа должна демонстрировать, что именно было было проделано её автором, 
я постарался использовать личное местоимение единственного числа в описании конкретных решений, 
результатов и выводов, сделанных самим автором работы и множественного числа при
изложении выводов формул и для указания результатов, полученных в соавторстве. 
Во всех остальных случаях был использован пассивный залог.\note{наверное эту всю эту фразу можно убрать}

Со всеми остальными замечаниями полностью согласен.
\end{document}

