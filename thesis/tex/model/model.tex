\documentclass{trlnotes}
\usepackage{trmath}
\addcompatiblelayout{thesis}
\setlayout{thesis}
\usepackage{trthm}
\usepackage{trsym} 
\usepackage{trphys}
\usepackage{silence}
% \usepackage{tikz}
\WarningFilter{latex}{Reference}
\usepackage[
backend=biber,
sorting=none,
natbib=true,
style=authoryear-comp,
language=russian
]{biblatex}
\addbibresource{../../thesisrefs.bib}
\graphicspath{{../../}}
\begin{document}
\setcounter{footnote}{0}
Одной из главных задач в численных расчётах является выбор начальных условий. Для того, чтобы изучать природу барлинз и B/PS балджей, необходимо определить, при каких значениях параметров модели они возникают. Для этого необходимо обратиться к результатам наблюдений реальных галактик.

Как уже было описано в предыдущей главе, одним из самых важных параметров, определяющих морфологию бара в положении <<плашмя>>,
является градиент кривой вращения в центральной области, тесно связанный с профилем плотности вблизи центра галактики. В многокомпонентных моделях галактик определённую роль в создании градиента скорости будут играть все компоненты. Следуя за \citet{salo2017}, я буду исследовать только влияние параметров классического балджа, ограничиваясь обычными в литературе NFW-подобными профилями тёмного гало (\cite{navarro1996}). Поскольку барлинзы скорее характерны для дисковых галактиках ранних типов \citep{laurikainen2011}, бедных газом, а B/PS-структуры, в целом, крайне редко встречаются в Scd галактиках \citep{erwin2017, li2017a}\note{В работе \cite{lutticke2000}, однако, описано исключение из этого правила}, можно ограничится бесстолкновительным $N$-body моделированием и не усложнять расчёты учётом 
диссипативной компоненты и сопутствующими ей физическими процессами (как, например, в \cite{athanassoula2015}). 

В работе \citet{salo2017} параметры классического балджа подбирались на основе предыдущих работ этих же авторов по 
фотометрической декомпозиции изображений галактик из инфракрасного (3.6\thinspace μm) обзора S${}^4$G (Spitzer Survey of 
Stellar Structure in Galaxies). \citet{salo2015} в своих моделях брали относительную массу балджа (по отношению к диску), равной  
$M_b/M_d = 0.08$ и $0.01$. При этом эффективный радиус балджа в обоих случаях составлял $0.07$ экспоненциального 
масштаба диска. Такие значения массы балджа могут отличаться примерно в три раза от более ранних оценок, 
встречающихся в литературе, в силу особенностей фотометрической декомпозиции, таких как учёт барлинз и B/PS 
балджей \citep{laurikainen2016a}, но при этом, например, в работе \citet{erwin2015} получены сходные результаты для 
композитных балджей, состоящих из классического балджа и псевдобалджа. Таким образом, оценка $M_b/M_d = 0.01 \div 
0.1$ согласуется с современными наблюдательными данными.

В данной работы были рассмотрены четыре модели с разными параметрами классического балджа при неизменности параметров остальных компонент.  Методом проб и ошибок было обнаружено, что подобранные параметры обеспечивают плавный переход между разной морфологией бара, что очень важно для изучения возникающих различий. 

За основу были взяты результаты $N$-body расчётов из работы 
\citet{smirnov2018}. % и к ним добавлены новые расчёты.
Модели включали в себя экспоненциальный диск:
\begin{equation}
  \label{eq:rho_disk} \rho_d(R,z) = \frac{M_d}{4\pi R_d^2 z_d} \cdot \exp(-R/R_d) \cdot \sech^2(z/z_d)\,,
\end{equation}
где $R$~"--- цилиндрический радиус, $M_d$~"--- полная масса диска, $R_d$, $z_d$~"--- радиальный и вертикальный 
масштабы диска, соответственно. Начальный профиль радиальной дисперсии скорости тоже экспоненциальный:
\begin{equation}
  \label{eq:sigma_disk} σ_R = σ_0 \cdot \exp (-R/R_σ), \quad R_σ = 2 R_d, 
\end{equation}
$σ_0$ определена так, чтобы параметр Тумре \citep{toomre1964} на 
радиусе $R_σ$ был равен некоторому заданному значению.  Точные 
значения параметров для всех моделей приведены в 
таблице~\ref{tab:modelpars}.
В дальнейшем изложении для удобства мы будем пользоваться системой 
единиц, в которой $R_d = 1$, $M_d = 1$, $G = 1$. Приняв $R_d = 
3.5\,\text{кпк}$, $M_d = 5\cdot 10^{10}\,\solmass$, получим единицу 
времени, равной $\sqrt{\frac{R_d^3}{GM_d}}\approx 13.81\, \text{млн. лет}$. Единица частоты в этом случае~"--- 
$163.95\,\text{км}/\text{с}/\text{кпк}$.  

Тёмное гало в моделях задаётся усечённым профилем плотности, похожим на профиль NFW (\cite{navarro1996}):
\begin{equation}
  \rho_h(r) = \frac{C_h}{(r/r_s)^{\gamma_0}\left((r/r_s)^\eta+1\right)^{(\gamma_\infty-\gamma_0)/\eta}} \,, \label{eq:NFW}
\end{equation}
где $r_s$~"--- радиальный масштаб гало, принятый равным 6, $η$ 
описывает форму профиля во внешней области, $γ_0$ и 
$γ_{\infty}$~"--- внутренний и внешний наклон профиля, 
соответственно, а $C_h$ задаёт полную массу гало $M_h$. Были 
приняты следующие значения параметров профиля гало:
$η = 4/9$, $γ_0 = 7/9$, $γ_∞ = 31/9$~"---  более пологий в центральной области и более крутой на периферии, чем 
оригинальный профиль Наварро-Фрэнка-Уайта. $C_h$ выбрано так, чтобы $M_h(r < 4R_d) = 1.5$. 

\begin{table}[htpb]
  \centering
  \caption{Параметры рассматриваемых $N$-body моделей. Приведены в порядке увеличения концентрации балджа.}
  \label{tab:modelpars}
  \vskip 2ex
  \begin{tabular}{cr|ccccccc}
  \toprule
     & Модель & $z_d$ & $Q$ & $M_h$ & $M_b$ & $r_b$ & $\del v_{cb}/\del r \, (r_b/2)$ & \\
  \midrule
     &X   & $0.05$ & $1.2$ & $1.5$ & $0$ &  &  &\\
     &Xbl & $0.05$ & $1.2$ & $1.5$ & $0.2$ & $0.2$ & $0.79$ &\\
     &BLx & $0.05$ & $1.2$ & $1.5$ & $0.1$ & $0.1$ & $1.57$\\
     &BL  & $0.05$ & $1.2$ & $1.5$ & $0.1$ & $0.05$ & $4.44$ &\\
  \bottomrule
  \end{tabular}
\end{table}

В три из четырёх моделей в качестве дополнительного потенциала был добавлен потенциал классического балджа, задаваемый профилем 
  Хернквиста\note{Этот профиль является хорошим приближением закона де Вокулёра.} \citep{hernquist1990}: \begin{equation}
  \rho_b(r) = \frac{M_b\, r_b}{2\pi\,r\,(r_b + r)^3} \,,
\end{equation}
где $M_b$~"---полная масса балджа, а $r_b$ задаёт его характерный размер.  Принятые значения параметров балджей 
приведены в таблице~\ref{tab:modelpars}.
Зная профиль плотности, можно получить потенциал:
\[
  Φ(r) = -\frac{M_b}{r+r_b},
\]
а затем модельную кривую вращения:
\[
  v_{cb}(r) = \sqrt{r \, \fder{Φ}{r}} = \sqrt{\frac{M_b\, r}{(r + r_b)^2}} \,. 
\]
Из кривой вращения оценим её градиент в центральных областях, например, на расстоянии $r_b/2$.
\[
\left. \fder{v_{cb}}{r} \right|_{r=r_b/2} = 
\left. \sqrt{\frac{M}{r}}\, \frac{r_b-r}{2\, (r_b+r)^2} \right|_{r=r_b/2} = 
\sqrt{\frac{M}{r_b^3}}\, \frac{\sqrt{2}}{9}
\]
Из последнего равенства видно, что именно средняя концентрация балджа определяет наклон кривой вращения, и, как 
следствие, морфологию бара.

В $N$-body расчётах особое внимание следует уделять влиянию численных эффектов. В частности, при малом количестве 
частиц сильно уменьшается время релаксации, что ведёт к динамической эволюции модели, которая не соответствует 
процессам в реальных галактиках. В моделях, используемых в данной работе, диск состоял из $N_d = 4\cdot 10^5$ 
частиц, а гало из $N_h=4.5\cdot 10^5$; обоснование этого выбора можно найти в \citet{smirnov2018}.  Равновесная 
модель, соответствующая описанным начальным условиям, была подготовлена с помощью скрипта \texttt{mkgalaxy} 
\citep{mcmillan2007a}. Длина сглаживания потенциала для диска и гало принята равной $ε_d = 3.7\cdot 10^{-3}$ 
($\approx 13 \text{пк}$) и $ε_h = 12.9\cdot 10^{-3}$ ($\approx 45 \text{пк}$) соответственно.  Эти значения согласуются с результатами статьи \citet{salo2017}. В ней авторы анализировали  изменения морфологии бара в зависимости от длины сглаживания и получили, что больших $ε_d$ барлинза замывается. Для корректного моделирования барлинзы необходимо $ε_d<0.05$. Значение, взятое в данной работе, соответствует этому критерию.

Эволюция модели изучалась с помощью интегратора \texttt{gyrfalcON} \cite{dehnen2002}, использующего комбинацию
tree-code метода и мультипольного разложения, с асимптотической сложностью алгоритма $\mathcal O(N)$. Этот код 
является частью свободного пакета \texttt{NEMO} \citep{teuben1995a}, предоставляющего набор утилит для $N$-body 
моделирования. В каждой из 4 моделей в первоначальном осесимметричном диске довольно быстро формируется бар. К моменту $t=400$ бар уже вырос в вертикальном направлении и медленно меняет свою амплитуду и скорость узора% (\ref{fig:4models}). 
Для изучения его орбитальной структуры в следующих главах мы будем использовать методы спектральной динамики, впервые описанные в работе \citet{binney1982}. Эти методы основаны на поиске доминирующих частот в Фурье-спектрах временных рядов координат частиц. При этом необходимо достаточно высокое временное разрешение. Оно было достигнуто пересчётом моделей на промежутке времени от $t=400$ до $t=500$ с очень маленьким шагом по времени, а также за счёт малости изменения посчитанных частот на этом промежутке. На основе оценок, приведённых в \cite{parul2020}, на рассматриваемых временах бар эволюционирует медленно, и характерная величина сдвига частот сравнима с ошибками определения частот на сетке, что и позволяет проводить относительно надёжный анализ.

\end{document}
% vim:wrapmargin=3 formatoptions=aw2tq
