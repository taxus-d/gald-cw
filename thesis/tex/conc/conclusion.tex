\documentclass{trlnotes}
\usepackage{trmath}
\addcompatiblelayout{thesis}
\setlayout{thesis}
\usepackage{trthm}
\usepackage{trsym} 
\usepackage{trphys}
\usepackage{silence}
% \usepackage{tikz}
\WarningFilter{latex}{Reference}
\graphicspath{{../../}}
\usepackage[
backend=biber,
sorting=none,
natbib=true,
style=authoryear-comp,
language=russian
]{biblatex}
\addbibresource{../../thesisrefs.bib}
\begin{document}

В работе был проведён анализ орбитального состава в плоскости диска для 4-x $N$-body моделей с барами,
отличающихся массой и характерным радиусом классического балджа.
В полном согласии с описанными в литературе результатами, по мере увеличения центральной концентрации
балджа изменяется морфология бара плашмя~"--- происходит плавный переход от арахисообразного бара к 
бару с барлинзой.

Для исследования структуры моделей, был использован метод анализа частот, впервые описанный в \citet{binney1982}.
Поскольку определяющие долгоживующую структуру орбиты должны быть квазипероидическими, их спектры должны состоять из
дискретного набора пиков. Для всех 4 млн. частиц звёздного диска в каждой из моделей были посчитаны доминирующие частоты
$ω_x, ω_y, ω_z, ω_R$ по временным рядам декартовых координат и цилиндрического радиуса. 
С помощью полученных значений в модели был выделен весь бар и разработана схема классификации орбитальных семейств в нём.
Сравнение населённости различных семейств в различных моделях показало, что арахисообразная морфология бара создаётся, в 
основном ящикообразными орбитами, находящимися во внутреннем резонансе Линдблада, в то время как основная роль в формировании
округлых очертаний бара с барлинзой принадлежит семействам, не поддерживающим бар.

В ходе исследования было выделено компактное семейство находящееся в центральных областях бара, наибольшее
влияние на движение звезд в котором оказывает потенциал балджа. На мой взгляд оно является ключевым компонентом барлинзы, без
которого невозможно получить характерную для неё форму изофот. Если проследить эволюцию составляющих её звёзд к началу модели,
то окажется что она формируется \emph{in situ}, и натекания вещества практически не происходит. Однако в моделях с учётом физики газа
(например, \cite{athanassoula2015}) центральная концентрация массы (и барлинза, соответственно) формируется из-за 
звездообразования в натекающем газе. Таким образом, связь результатов этой работы с процессами, происходящими в реальных галактиках, такими
как постепенное образование псевдобалджа и разрушение бара до состояния линзы \citep{combes2011} ещё предстоит установить.
\end{document}
% vim:wrapmargin=3
