\documentclass{trlnotes}
\usepackage{trmath}
\addcompatiblelayout{thesis}
\setlayout{thesis}
\usepackage{trthm}
\usepackage{trsym} 
\usepackage{trphys}
\usepackage{silence}
% \usepackage{tikz}
\WarningFilter{latex}{Reference}
\graphicspath{{../../}}
\usepackage[
backend=biber,
sorting=none,
natbib=true,
style=authoryear-comp,
language=russian
]{biblatex}
\addbibresource{../../thesisrefs.bib}
\begin{document}

В работе был проведён анализ орбитального состава диска для 4-x $N$-body моделей с барами,
отличающихся массой и характерным радиусом классического балджа. Основное внимание было уделено морфологии орбит в плоскости диска. 
В полном согласии результатами, описанными в литературе, я нашёл, что по мере увеличения центральной концентрации балджа изменяется морфология бара плашмя~"--- происходит плавный переход от арахисообразного бара к 
бару с барлинзой.

Для исследования орбитальной структуры изучаемых моделей, был использован метод анализа частот, впервые описанный в \citet{binney1982}.
Поскольку орбиты, определяющие долгоживующую структуру в звёздном диске, должны быть квазипероидическими, их спектры должны состоять из
дискретного набора пиков. Для всех 4 млн. частиц звёздного диска в каждой из моделей были посчитаны доминирующие частоты
$ω_x, ω_y, ω_z, ω_R$ по временным рядам декартовых координат и цилиндрического радиуса. Работа такого объема для исследования морфологии орбит в плоскости диска ещё никем до этого не проводилась. Это делается впервые. 
С помощью полученных значений частот и их отношений в исследуемых моделях был выделен весь бар. Далее была разработана схема классификации орбитальных семейств в нём.
Сравнение населённости различных семейств в различных моделях показало, что арахисообразная морфология бара создаётся, в 
основном ящикообразными орбитами, находящимися во внутреннем резонансе Линдблада. Это подтверждает результаты предыдущих работ, основанных, как на анализе большого числа орбит в <<замороженных>> потенциалах, так и на анализе скромного числа выборочных орбит в $N$-body потенциалах. 

Из новых, ранее никем не полученных результатов отмечу следующие.

1. Сделан вывод, что основная роль в формировании округлых очертаний бара с барлинзой принадлежит семействам, не поддерживающим бар. Если до этого были только качественные рассуждения о возможных типах орбит, населяющих барлинзу, то теперь и сама структура, и семейства орбит, её поддерживающие, выделены непосредственно в моделях.

2. В ходе исследования было выделено компактное орбитальное семейство, находящееся в центральных областях бара.  На движение звёзд, принадлежащих этому орбитальному семейству, наибольшее влияние оказывает потенциал балджа. Именно он создаёт условия для рассеяния орбит в центральных областях и препятствует захвату быстропрецессирующих орбит в бар. На мой взгляд, это семейство является ключевым компонентом барлинзы, без
которого невозможно получить характерную для неё округлую форму изофот, наблюдающуюся у многих галактик с барлинзой. Само семейство с точки зрения формы орбит очень простое, но оно ещё ни разу не выделялось в подобного рода исследованиях. Его орбитальный состав открывает большие возможности для создания наблюдательных кинематических тестов и дальнейшего изучения таких интересных объектов, как галактики с барлинзами.

3. Ещё одним интересным результатом исследования является вывод, что наблюдаемые в галактиках с барлинзами следы X-образных структур на изображениях с нерезким маскированием могут говорить о том, что структура барлинзы определяется тонким балансом между населённостью различных семейств, в частности, ролью <<ящикообразного>> коротного бара. Об этом же говорит существование галактик с барлинзами, имеющими не столько округлые, сколько ящикообразные изофоты. В этом случае в формировании таких изофот на первый план может выходить протяженное семейство $\text{bl}_{\text{o}}$

4. В исследованных моделях мне удалось также выделилить центральное семейство в баре, перепендикулярное основному вытянутому бару. Оказалось, что для моделей без компактного балджа это семейство $x_4$. Оно, правда, малочисленно. В моделях с компактными балджами --- это семейство $x_2$, и его населённость увеличивается по мере увеличения компактности балджа. Это семейство может отвечать за внутренние перпендикулярные бары, часто наблюдающиеся в галактиках с барлинзами.

В заключении я хочу возвратиться к орбитальному семейству, которое, по-видимому, является основным для формирования округлой формы барлинзы. 
Если проследить эволюцию этой структуры, создаваемой этим семейством, отмотав её по времени к самому началу эволюции модели, то окажется, что она формируется \emph{in situ}, и натекания вещества в эту структуру из других областей практически не происходит. Однако в моделях с учётом физики газа (например, \cite{athanassoula2015}) центральная концентрация массы и <<будущая>> барлинза, соответственно, формируются из-за 
звездообразования в натекающем на центр газе. Таким образом, связь результатов этой работы с процессами, происходящими в реальных галактиках, такими
как постепенное образование псевдобалджа и разрушение бара до состояния линзы \citep{combes2011} ещё предстоит установить. Но эта связь может оказаться важной только для галактик более поздних типов, с газом. К галактикам ранних типов результаты моей работы могут быть приложены напрямую.
 
\end{document}
% vim:wrapmargin=3
