\documentclass{trlnotes}
\usepackage{trmath}
\addcompatiblelayout{thesis}
\setlayout{thesis}
\usepackage{trthm}
\usepackage{trsym} 
\usepackage{trphys}
\usepackage{silence}
% \usepackage{tikz}
\usepackage[
backend=biber,
sorting=none,
natbib=true,
style=authoryear-comp,
language=russian
]{biblatex}
\addbibresource{../../thesisrefs.bib}
\graphicspath{{../../}}
\begin{document}

Существование различимых структур в модели, сохраняющихся на масштабах $10^9$ лет
возможно лишь при наличии стабильных околопериодических орбит, определяющих их морфологию. Такие орбиты должны
лежать в недалеко друг от друга в фазовом пространстве. Мы будем называть их орбитальными семействами. 

Более точное определение семейства зависит от рассматриваемого представления фазового пространства, такого,
например, как сечения Пуанкаре, используемого в классической литературе посвящённой анализу орбит в модельных
потенциалах в двумерном случае \citep{contopoulos1980a,2008gady.book.....B}. Однако такой подход плохо подходит
для изучения $N$-body моделей~"--- орбиты частиц трёхмерные и их энергии могут принимать широкий диапазон
значений. Оба этих фактора приводят к <<размытию>> сечений Пуанкаре и усложняют задачу
классификации \citep{valluri2016}. Поэтому в настоящей работе используется другой метод представления фазового
пространства~"--- распределения отношений доминирующих частот (frequency map). 

Попробуем разобраться в физическом смысле такого распределения. Перейдём в координаты действие-угол: $(J_i,
\theta_i)$, тогда из определения интеграла движения и уравнений Гамильтона получим 
\[
  0 = \dot J_i = -\pder{H}{θ_i} \so \dot θ_i = \pder{H}{J_i} = Ω_i(\v J) = \text{const} \so θ_i(t) = θ_i^0 +
  Ω_i(t)
\]
Ограничиваясь замкнутыми орбитами (все остальные быстро покидают систему и не могут образовывать долгоживущую
структуру), можно записать разложение декартовых координат в ряд Фурье
\[
  x_i (t, \v J) = \sum_{\v n\in \Z^m} X_{i,n} (\v J)\, e^{i\,\mathbf{n}\cdot \v θ^0}\, e^{i\,\v n\cdot \v Ω\,t}\,,
\]
определяемый набором \emph{фундаментальных} частот: $\{Ω_i(\v J)\}_i$. Орбита называется \emph{резонансной}, если
её фундаментальные частоты соизмеримы: $\exists\, n \neq 0\that \v n \cdot \v Ω = 0$. В координатах действие-угол
траектории частиц <<наматываются>> на торы, определяемые интегралами движения частиц (например, полная энергия),
при этом нерезонансные орбиты заметают весь тор, а квазипериодические~"--- ограниченную область на нем. Это
фактически означает, что для них есть свой интеграл, который понижать размерность резонансного семейства в фазовом
пространстве\quest. Таким образом, резонансные орбиты структурируют фазовое пространство, и представляется
естественным классифицировать орбиты, опираясь на эту структуру.

Подобный подход тоже часто встречается в литературе, начиная с классической работы \citet{binney1982} и
включая более современные \citep{athanassoula2002,portail2015,valluri2016} и подробно описан в обзоре
\citet{athanassoula2013}. Общая идея частотного анализа состоит в поиске пиков на спектрах координат и
цилиндрического радиуса $R = \sqrt{x^2 + y^2}$, из положений которых вычисляются исследуемые частоты.
При этом, чаще всего изучаются орбиты либо в аналитических потенциалах, либо в <<замороженных>> на какой-то
фиксированный момент времени потенциалах из $N$-body моделирования. В обоих случаях потенциалу искусственно
придаётся вращение с угловой скоростью бара. Необходимость этого обусловлена требованием независимости интегралов
движения от времени, что может не выполняться в самосогласованной $N$-body на больших промежутках времени. Однако,
в \citet{ceverino2007} был применён другой подход~"--- авторы изучали орбиты на таком интервале времени, где
потенциал галактики и скорость вращения бара изменяются незначительно. При этом пики на спектрах размываются, но
полученные результаты качественно совпадают с \cite{athanassoula2002a}.
Мы будем использовать сходную методику, во многом следуя \citet{gajda2016}, поскольку её результаты скорее всего
ближе к структуре реальных галактик.

Орбитальное движение частиц в модели рассматривается в системе отсчёта, вращающейся вместо с баром. Ось $x$
направлена вдоль его большой оси, $z$ перпендикулярна плоскости диска, а $y$ дополняет их до правой тройки.  Для
каждой из $4\cdot 10^6$ частиц строятся временные ряды $\{x_k(t), y_k(t), z_k(t), R_k(t)\}$ от $t_1=400$ (5.3
млрд. лет) до $t_2=500$ (6.6 млрд. лет) с шагом $ΔT = 0.125$ единиц, после чего, с помощью кода, написанного
Смирновым~А.А. \plholdev{как сослаться правильно}\quest{} вычисляются частоты, соответствующие пикам на
периодограмме с наибольшей амплитудой. Вычитая из изначального ряда сигнал, модулированный основной частотой и
вычисляя периодограмму разности, можно определить второй по величине пик и так далее. 

Из-за дискретности временного ряда, максимальное возможное значение частоты ограничено частотой Найквиста: 
\[
  ν_c = \frac{1}{2 Δt} = 4, \quad ω_c = 2\pi ν_c \approx 4120\,\text{км}/\text{с}/\text{кпк}\,,
\]
a ограниченная длина временного ряда приводит к ограниченному разрешению по частотам: 
\[
  Δν = \frac{1}{t_2 - t_1} = 0.01, \quad δω = 2\pi Δν \approx 10\,\text{км}/\text{с}/\text{кпк}.
\]
При этом, вторая величина определяет точность определения пиков на спектре и для аккуратного выделения резонансных
орбит её не достаточно. Используемый алгоритм решает эту проблему, находя сначала положения пиков на изначальной
сетке частот, а потом пересчитывая преобразование Фурье в окрестности пика с шагом $δ_1ν = 0.001$. Являясь
разновидностью дополнения временного ряда нулями, такой способ не позволяет различить сливающиеся на грубой сетке
пики, однако десятикратно увеличивает точность определения положения одиночного пика. Более полно детали алгоритма
и оценка дрейфа частот описаны в \citet{parul2020}, для используемых в этой работе моделей на заданном временном
интервале его можно принять меньшим $0.01$~"--- разрешения изначальной сетки.

В следующей главе будет подробно описан процесс выделения различных орбитальных семейств с помощью отношений
полученных частот ($f_x, f_y, f_z, f_R$) для всех моделей. Перед этим важно отметить, что используемый метод не
позволяет \emph{точно} разделить резонансные и близкие к ним но хаотические орбиты просто в силу ограниченного
разрешения по частоте. Поэтому, в одно семейство мы будем выделять сходные по морфологии орбиты с достаточно
близкими отношениями частот, используя при этом всю доступную информацию, например амплитуды соответствующих пиков
периодограммы. Подобная проблема не так выражена в модельных или замороженных потенциалах, не подверженных
секулярной эволюции\note{возможно я использую термин секулярный неправильно!}\quest, однако исследовать их не так
интересно.  

\end{document}
% vim:wrapmargin=3
