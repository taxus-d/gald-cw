\documentclass{trlnotes}
\usepackage{trmath}
\addcompatiblelayout{thesis}
\setlayout{thesis}
\usepackage{trthm}
\usepackage{trsym} 
\usepackage{trphys}
\usepackage{silence}
% \usepackage{tikz}
\usepackage[
backend=biber,
sorting=none,
natbib=true,
style=authoryear-comp,
language=russian
]{biblatex}
\addbibresource{../../thesisrefs.bib}
\graphicspath{{../../}}
\begin{document}

Существование структур в моделях, сохраняющихся на масштабах времени $10^9$ лет, возможно лишь при наличии стабильных околопериодических орбит. Эти орбиты и определяют  общую морфологию структур. Такие орбиты должны лежать недалеко друг от друга в фазовом пространстве. Мы будем называть их орбитальными семействами. 

Более точное определение семейства зависит от рассматриваемого представления фазового пространства. В классической литературе, посвящённой анализу орбит в модельных
потенциалах в двумерном случае \citep{contopoulos1980a,2008gady.book.....B}, в качестве фазового пространства используют чаще всего сечения Пуанкаре. Однако такой подход плохо подходит для изучения $N$-body моделей~"--- орбиты частиц трёхмерные, и их энергии могут принимать значения в широком диапазоне. Оба этих фактора приводят к <<размытию>> сечений Пуанкаре и усложняют задачу
классификации орбит \citep{valluri2016}. Поэтому в настоящей работе используется другой метод представления фазового
пространства~"--- двумерные распределения отношений доминирующих частот (frequency map). 

Попробуем разобраться в физическом смысле такого распределения. Перейдём к координатам действие-угол: $(J_i,
\theta_i)$. Тогда из определения интеграла движения и уравнений Гамильтона получим 
\[
  0 = \dot J_i = -\pder{H}{θ_i} \so \dot θ_i = \pder{H}{J_i} =
  Ω_i(\v J) = \text{const} \so θ_i(t) = θ_i^0 + Ω_i \,t
\]
Ограничиваясь замкнутыми орбитами (все остальные быстро покидают систему и не могут образовывать долгоживущую
структуру), можно записать разложение декартовых координат в ряд Фурье
\[
  x_i (t, \v J) = \sum_{\v n\in \Z^m} X_{i,n} (\v J)\, e^{i\,\mathbf{n}\cdot \v θ^0}\, e^{i\,\v n\cdot \v Ω\,t}\,,
\]
определяемый набором \emph{фундаментальных} частот: $\{Ω_i(\v J)\}_i$. Орбита называется \emph{резонансной}, если
её фундаментальные частоты соизмеримы: $\exists\, n \neq 0\that \v n \cdot \v Ω = 0$. В координатах действие-угол
траектории частиц <<наматываются>> на торы, задаваемые интегралами движения частиц (например, полная энергия, угловой момент), при этом нерезонансные орбиты заметают весь тор, а квазипериодические~"--- ограниченную область на нем. Это
фактически означает, что для них есть свой интеграл, который понижает размерность резонансного семейства в фазовом
пространстве. Таким образом, резонансные орбиты структурируют фазовое пространство, и представляется
естественным классифицировать орбиты, опираясь на эту структуру в фазовом пространстве как ``скелет'' структуры в обычном пространстве.

Подобный подход часто встречается в литературе. Можно сослаться на классическую работу \citet{binney1982} и более
современные работы  \citep{athanassoula2002,portail2015,valluri2016}. Подробно этот подход описан в обзоре
\citet{athanassoula2013}. Общая идея частотного анализа состоит в поиске пиков на спектрах координат и цилиндрического
радиуса $R = \sqrt{x^2 + y^2}$. Положение пиков используется для  вычисления исследуемых частот.  При этом, чаще всего
изучаются орбиты либо в аналитических потенциалах, либо в потенциалах из $N$-body моделирования, но <<замороженных>> на
какой-то фиксированный момент времени. В обоих случаях потенциалу искусственно придаётся вращение с угловой скоростью
бара. Необходимость этого обусловлена требованием независимости интегралов движения от времени, что может не
выполняться в самосогласованных $N$-body расчётах на больших промежутках времени.  Однако, в
работе \citet{ceverino2007}
был применён другой, более трудоёмкий, но более <<честный>> подход~"--- авторы изучали орбиты непосредственно в
<<живом>> $N$-body потенциале, но на таком интервале времени, где потенциал галактики и скорость вращения бара
изменяются незначительно. При этом пики на спектрах естественно размываются, но полученные авторами результаты качественно совпадали с результатами стандартного подхода \citep{athanassoula2002a}.  Мы будем использовать сходную методику, во многом следуя работе \citet{gajda2016}. Результаты этой работы, по-видимому, ближе всего соответствуют структуре реальных галактик.

Орбитальное движение частиц в модели рассматривается в системе отсчёта, вращающейся вместо с баром. Ось $x$ направлена
вдоль большой оси бара, $z$ перпендикулярна плоскости диска, а $y$ дополняет их до правой тройки.  Для каждой из $4\cdot
10^6$ частиц диска строятся временные ряды $\{x_k(t), y_k(t), z_k(t), R_k(t)\}$ от $t_1=400$ (5.3 млрд. лет) до
$t_2=500$ (6.6 млрд. лет) с шагом $Δt = 0.125$ единиц, после чего, с помощью кода, написанного аспирантом СПбГУ
Смирновым~А.А., вычисляются частоты, соответствующие пикам на периодограмме с наибольшей амплитудой. Вычитая из
изначального ряда сигнал, модулированный основной частотой, и вычисляя периодограмму разности, можно определить второй по
величине пик и так далее. Всего выделялось три пика, но анализ орбитальных семейств в основном производился по первым
пикам, с привлечением вторых.

\begin{figure}[htpb]
  \centering
  \includepgfgraphics[height=7cm]{img/method/orbit-3793601-xy.pgf}
  \includepgfgraphics[height=7cm]{img/method/orbit-3793601-timeseries.pgf}
  \includepgfgraphics[height=7cm]{img/method/orbit-3793601-spectra.pgf}
  \caption{Пример спектра для одной из орбит вблизи центра модели. Сверху: временные ряды $x(t)$, $y(t)$, $R(t)$ на промежутке
  $[400; 420]$, снизу: соответствующие им периодограммы. Промежуток времени выбран так, чтобы отдельные кривые были
  различимы на графике, вычисление частот идёт на всем доступном промежутке $[0;100]$.
  }%
  \label{fig:spectrasample}
\end{figure}

Из-за дискретности временного ряда, максимальное возможное значение частоты ограничено частотой Найквиста: 
\[
  ν_c = \frac{1}{2 Δt} = 4, \quad ω_c = 2\pi ν_c \approx 4120\,\text{км}/\text{с}/\text{кпк}\,,
\]
Однако, даже для орбит вблизи центра модели, вычисленные значения частот много меньше частоты Найквиста
и алиасинга не происходит (рис.~\ref{fig:spectrasample}).

Ограниченная длина временного ряда приводит к ограниченному разрешению по частотам: 
\[
  Δν = \frac{1}{t_2 - t_1} = 0.01, \quad Δω = 2\pi Δν \approx 10\,\text{км}/\text{с}/\text{кпк}.
\]
При этом, вторая величина определяет точность определения пиков на спектре, и для аккуратного выделения резонансных
орбит её не достаточно.  Используемый алгоритм решает эту проблему следующим образом. Сначала находятся положения пиков
на изначальной сетке частот, а потом пересчитывается преобразование Фурье в окрестности пика с шагом $Δ_1ν = 0.001$, т.е. в 10 раз меньшим, чем изначальный.
Являясь разновидностью схемы дополнения временного ряда нулями, такой способ не позволяет различить сливающиеся на
грубой сетке пики, однако десятикратно увеличивает точность определения положения одиночного пика. Более полно детали
алгоритма и оценка дрейфа частот описаны в \citet{parul2020}. Для моделей, анализируемых в моей работе, на заданном
временном интервале скорость дрейфа меньше, чем $0.01$~"--- разрешение изначальной сетки.

В следующей главе будет подробно описан процесс выделения различных орбитальных семейств с помощью отношений полученных
частот ($ω_x, ω_y, ω_z, ω_R$) для всех моделей. Перед этим важно отметить, что используемый метод не позволяет
\emph{точно} разделить резонансные, квазипериодические и близкие к ним, но хаотические орбиты просто в силу
ограниченного разрешения по частоте. Поэтому, в одно семейство мы будем выделять сходные по морфологии орбиты с
довольно близкими отношениями частот, используя при этом всю доступную информацию, например амплитуды соответствующих
пиков периодограммы. Подобная проблема не так выражена в модельных или замороженных потенциалах, не подверженных
долговременной эволюции, однако исследования таких потенциалов носят всё же академический характер и не дают полного
понимания орбитальной структуры <<живой>> модели.

\end{document}
% vim:wrapmargin=3
