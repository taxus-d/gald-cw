\documentclass[tikz]{trlnotes}
\usepackage{trmath}
\addcompatiblelayout{thesis}
\setlayout{thesis}
\usepackage{trthm}
\usepackage{trsym} 
\usepackage{trphys}
\usepackage{silence}
% \usepackage{tikz}
\WarningFilter{latex}{Reference}
\graphicspath{{../../}}
\usepackage[
backend=biber,
sorting=none,
natbib=true,
style=authoryear-comp,
language=russian
]{biblatex}
\addbibresource{../../thesisrefs.bib}
\begin{document}
Попробуем подвести итоги анализа орбитальных семейств, проведённого в двух предыдущих главах,
и разобраться в причинах изменения морфологии.  Два уже отмеченных ранее результата видны из
таблицы~\ref{tab:familiesnumbers}: по мере увеличения концентрации балджа в модели уменьшается количество орбит в
boxy bar семействе и увеличивается доля не поддерживающих бар семейств.
Первый эффект хорошо заметен на рис.~\ref{fig:boxyamplcomp}, где приведены распределения средних протяжённостей
орбит в boxy bar семействе для всех четырёх моделей. Эта иллюстрация подчёркивает и другой довольно важный
тренд~"--- постепенное уменьшение размера ящикообразной орбиты при переходе от X к BL, связанное
с уменьшением радиуса коротации и длины самого бара. При этом, не поддерживающие бар семейства (например
$\text{bl}_{\text{o}}$, остаются примерно постоянного размера и растут в численности. Из-за этого boxy bar семейство
по мере перехода к модели с барлинзой <<тонет>> в остальных орбитах, и арахисообразная структура перестаёт быть
видимой, уступая место округлой.

Не менее интересна и вторая тенденция. На рис.~\ref{fig:axisratio}, где приведены одномерные распределения
отношения средней протяжённости по двум осям, заметно как от арахисообразного бара к бару с барлинзой 
постепенно увеличивается высота пика на $\averg{|y|}/\averg{|x|}$, соответствующего орбитам, вытянутым вдоль оси
$y$, и не поддерживающим бар. Раз их доля растёт, количество орбит в ILR уменьшается и бар становится слабее и
меньше. 

\begin{figure}[htpb]
  \centering
  \includepgfgraphics[width=.9\textwidth]{img/discuss/boxyampcomp.pgf}
  \caption{Одномерные распределения средней протяжённости по оси $x$ членов семейства boxy bar с шириной бина $0.01$.
  По мере увеличения компактности балджа падает количество орбит в нём и их средняя протяжённость.}
  \label{fig:boxyamplcomp}
\end{figure}

\begin{figure}[htpb]
  \centering
  \includepgfgraphics[width=.9\textwidth]{img/discuss/axisratio.pgf}
  \caption{Одномерные распределения отношения средней протяжённости по оси $y$ к средней протяжённости по оси $x$ для орбит из
    области, содержащей бар, с шириной бина $0.01$. По мере увеличения компактности балджа появляются становится заметен вклад орбит,
    вытянутых вдоль оси $y$.}
  \label{fig:axisratio}
\end{figure}

Движение орбит вблизи центра модели в большей степени определяется потенциалом балджа, чем бара. 
Чтобы убедиться в этом, обратимся к рис.~\ref{fig:clres}, на котором изображены зависимости $ω_x - ω_R/m$,
$m=-2,0,2$ от среднего радиуса орбиты. В частности, поскольку $ω_x \approx Ω - Ω_p$, прямая $ω_x - ω_R/2 = 0$
должна соответствовать бару (ILR) и, как хорошо видно на этой иллюстрации, простирается до коротации~"--- точки
пересечения $ω_x$ с осью абсцисс. Орбиты, лежащие выше этой прямой принадлежат к семейству $\text{bl}_{\text{u}}$,
а под ней~"--- к $\text{bl}_{\text{o}}$. Чёрная линия на этом графике показывает аналитическую зависимость 
$Ω - Ω_p - ϰ/2$ для потенциала Хернквиста (в плоскости диска):
\begin{equation}
  Ω^2 = \frac{1}{R}\pder{Φ}{R} = \frac{M}{R}\,\frac{1}{(R+r_b)^2}, \quad ϰ^2 = \pder[2]{Φ}{R} + 3 Ω^2 =
\frac{M}{R}\, \frac{R +3r_b}{(R+r_b)^3}.
\end{equation}
Отмечу, что члены семейства $\text{bl}_{\text{u}}$ достаточно точно следуют модельной кривой. Это означает, что
они <<чувствуют>> в основном потенциал балджа, из-за чего их скорости прецессии не синхронизуются с
орбитами в баре, и во вращающейся вместе с баром системе отсчёта возникает довольно характерная округлая морфология. 

\begin{figure}
  \centering
  \includepgfgraphics[width=.9\textwidth]{img/discuss/classicalresonances.pgf}
  \caption{Двумерные распределения $|ω_x| - |ω_R|/m$ от $\averg{R}$, $m = -2,0,2$ для модели BL.
    Чёрная линия соответствует $Ω(R) - Ω_p - ϰ(R)/2$ в потенциале балджа.
  Пересечение жёлтой кривой с осью абсцисс соответствует ILR, синей кривой~"--- коротации. 
  За коротацией знак $ω_x$ становится отрицательным, поэтому проследить положение OLR на этом графике невозможно.}
  \label{fig:clres}
\end{figure}


% \paragraph{Доминирующие частоты в эпициклическом приближении}
При рассмотрении орбит в потенциале галактики часто используют эпициклическое приближение, сводящее движение
звезды по орбите как движения по паре окружностей, центр меньшей из которых движется по большей. В N-body
моделях с выраженным баром, как правило, такое приближение не работает. Однако, как мы убедились чуть раньше, для
частиц в моделях с балджем именно его потенциал играет определяющую роль. Поэтому, для изучения их движения бар
можно рассмотреть как возмущение, что позволяет воспользоваться эпициклическим приближением
\citep[стр.~189]{2008gady.book.....B}.

Положим
\begin{equation}
  \begin{aligned}
    x &= A_x \cos (\omega_x t + φ_x) + A_x^{(1)} \cos(\omega_x^{(1)} t + φ_x^{(1)}), & A_x &> A_x^{(1)} \\
    y &= A_y \cos (\omega_y t + φ_y) + A_y^{(1)} \cos(\omega_y^{(1)} t + φ_y^{(1)}), & A_x &> A_x^{(1)}, \\
  \end{aligned}
\end{equation}
где $φ$~"--- начальные фазы. Для краткости, мы будем опускать нулевой индекс у частот, фаз и амплитуд.
Рассмотрим, какие линии будут возникать в спектрах координат и цилиндрического радиуса. 

Из предыдущего уравнения, введя для удобства $ψ_k^{(j)} = ω_k^{(j)}t + φ_k^{(j)}$,
нетрудно получить выражение для квадрата цилиндрического радиуса
\begin{equation}
  \label{eq:xsqysqexpansion}
  \begin{split}
    x^2 + y^2  &= \frac{A_x^2 + \left(A_x^{(1)}\right)^2 + A_y^2 + \left(A_y^{(1)}\right)^2}{2} \\
               &+ \frac{A_x^2}{2} \cos 2ψ_x + \frac{\left(A_x^{(1)}\right)^2}{2} \cos 2ψ_x^{(1)} \\
         &+ \frac{A_y^2}{2} \cos 2ψ_y + \frac{\left(A_y^{(1)}\right)^2}{2} \cos 2ψ_y^{(1)} \\
         &+ A_x A_x^{(1)} \, \left( \cos (ψ_x + ψ_x^{(1)}) + \cos (ψ_x - ψ_x^{(1)}) \right) \\
         &+ A_y A_y^{(1)} \, \left( \cos (ψ_y - ψ_y^{(1)}) + \cos (ψ_y + ψ_y^{(1)}) \right). \\
  \end{split}
\end{equation}
Применим к получившемуся выражению преобразование Фурье. Поскольку оно аддитивно, достаточно разобраться только
с одним слагаемым:
\begin{equation}\label{eq:fftcos}
  \mathcal{F}\left[A_k^{(j)} \cos (ω_k^{(j)}t + φ_k^{(j)})\right] = 
  A_k^{(j)}\pi \, e^{iφ_{k}^{(j)}}\,δ(ω-ω_k^{(j)}) + A_k^{(j)}\pi\,e^{-iφ}\,δ(ω + ω_k^{(j)})
\end{equation}
Предполагая, что все аргументы $δ$ различны, получим набор дискретных пиков, приведённых ниже вместе с
соответствующими амплитудами\note{здесь и далее поделёнными на $2\pi$}:
\begin{equation}
  \label{eq:twooscpeaks}
  \begin{aligned}
    &\frac{A_x^2}{2}, & & \pm 2\omega_x;\quad
    &\frac{\left(A_x^{(1)}\right)^2}{2}, & & \pm 2\omega_x^{(1)};\quad
    &&A_x A_x^{(1)}, & & \pm \omega_x \pm \omega_x^{(1)}; \\
    &\frac{A_y^2}{2}, & & \pm 2\omega_y;\quad
    &\frac{\left(A_y^{(1)}\right)^2}{2}, & & \pm 2\omega_y^{(1)};\quad
    &&A_y A_y^{(1)}, & & \pm \omega_y \pm \omega_y^{(1)}.
  \end{aligned}
\end{equation}
Амплитуда пика в нуле мне не потребуется, поэтому для краткости опущена.
Фазы в эти выражения не входят, поскольку $|e^{iφ}|=1$.

Полагая, что $\mathrm{R} = \sqrt{x^2 + y^2}$ может быть представлен в виде ряда Фурье в виде
\begin{equation}
  R = A + A_R \cos (\omega_R t + φ_R) + \sum_{k \geqslant 1} A_R^{(k)}\, \cos (\omega_R^{(k)} t + φ_R^{(k)}),
  \quad A > A_R >A_R^{(1)} \dotsb,
\end{equation}
можно получить расположение самых высоких пиков на его спектре (снова опуская амплитуду в нуле):
\begin{equation}
  \begin{aligned}
    &\frac{\left(A_R^{(k)}\right)^2}{2},& & \pm 2\omega_R^{(k)}; \quad
    &A\,A_R^{(k)}, && \pm \omega_R^{(k)}                       ; \quad
    &A_R^{(k)} A_R^{(m)},&& \pm \omega_R^{(k)} \pm \omega_R^{(m)}.
  \end{aligned}
\end{equation}
Из наложенных ограничений на амплитуды вытекает, что высочайший пик на спектре радиуса расположен на $\omega_R$.
При этом, его положение должно быть согласовано с одним из значений в \eqref{eq:twooscpeaks}.

Вернёмся к описанию орбит их $\text{bl}_\text{u}$ и их спектров. В главе~\ref{chap:orbits} было показано, что на
их спектрах хорошо видна вторая линия, сравнимая по высоте с доминирующей. Это означает, что на спектрах $R(t)$
для таких орбит главный пик может попасть не в $2\,ω_x$, а в сумму: $ω_x + ω_{x}^{(1)}$. Анализируя распределения
отношения $ω_x^{(1)}/ω_x$ и $ω_y^{(1)}/ω_y$ для семейства $\text{bl}_\text{u}$, я обнаружил что для большей части
его членов $ω_x = ω_y$ (рис.~\ref{fig:fRxfyx_bar}), $ω_x^{(1)} = ω_y^{(1)}$, и
$ω_x^{(1)} < ω_x$. Таким образом, сумма $ω_R = ω_x + ω_x^{(1)} < 2\, ω_x$ и орбита не попадает в бар. 
Обсудим теперь условия, которое должно быть выполнено для вторых амплитуд в этом случае.
Поскольку некоторые из аргументов $δ$ совпадают, потребуется более аккуратный расчёт с привлечением
фаз, полученных при анализе доминирующих частот. Для семейства $\text{bl}_\text{u}$ большая часть орбит имеет
$φ_y - φ_x = \frac{\pi}{2}$, $φ_y^{(1)} - φ_x^{(1)} = -\frac{\pi}{2}$. Отсюда можно вывести точные выражения для
амплитуд пиков.
% если вдруг потребуется более подробная выкладка
% \begin{align}
%   ω &= 2ω_x = 2ω_y\,, & A &=\tfrac 12 \left|A_x^2 + A_y^2 \, e^{2i\, (φ_y-φ_x)}\right| =
%   \tfrac 12 \left|A_x^2 - A_y^2\right| \\
%   ω &= 2ω_x^{(1)} = 2ω_y^{(1)}\,, & A &=\tfrac 12 \left|
%     \left(A_x^{(1)}\right)^2 + \left(A_y^{(1)}\right)^2 e^{2i\, (φ_y^{(1)} - φ_x^{(1)})}
%   \right| = \tfrac 12 \left| \left(A_x^{(1)}\right)^2 - \left(A_y^{(1)}\right)^2\right| \\
%   ω &= ω_x + ω_x^{(1)} \,, & A &= \left|
%     A_xA_x^{(1)} + A_yA_y^{(1)} e^{i\, (φ_y + φ_y^{(1)} - φ_x - φ_x^{(1)})}
%   \right| = \left|A_xA_x^{(1)} + A_yA_y^{(1)} \right|\\
%   ω &= ω_x - ω_x^{(1)} \,, & A &= \left|
%     A_xA_x^{(1)} + A_yA_y^{(1)} e^{i\, (φ_y - φ_y^{(1)} - φ_x + φ_x^{(1)})}
%   \right| = \left|A_xA_x^{(1)} - A_yA_y^{(1)} \right|
% \end{align}
\begin{subequations}
\begin{align}
  ω &= 2ω_x = 2ω_y\,, & A &= \tfrac 12 \left|A_x^2 - A_y^2\right| \\
  ω &= 2ω_x^{(1)} = 2ω_y^{(1)}\,, & A &= \tfrac 12 \left| \left(A_x^{(1)}\right)^2-\left(A_y^{(1)}\right)^2\right|\\
  ω &= ω_x + ω_x^{(1)} \,, & A &= \left|A_xA_x^{(1)} + A_yA_y^{(1)} \right|\label{eq:freqsum}\\
  ω &= ω_x - ω_x^{(1)} \,, & A &= \left|A_xA_x^{(1)} - A_yA_y^{(1)} \right|
\end{align}
\end{subequations}
Для выполнения условия $ω_R = ω_x + ω_x^{(1)}$, необходимо чтобы амплитуда, задаваемая~\eqref{eq:freqsum}, была
больше всех остальных. Единственное нетривиальное неравенство, которое при этом возникает, следующее:
\begin{equation}
  A_xA_x^{(1)} + A_yA_y^{(1)} > \tfrac{1}{2}\,\left|A_x^2 - A_y^2\right| \iff
  2\frac{A_x^{(1)}}{A_x} + 2\frac{A_y}{A_x}\,\frac{A_y^{(1)}}{A_x} > \left|1 -
  \left(\frac{A_y}{A_x}\right)^2\right|.
\end{equation}
Для его выполнения достаточно $A_x^{(1)} > A_x/2$, что верно для большей части семейства $\text{bl}_\text{u}$.

Отметим, что для орбит с $ω_y = ω_x^{(1)}$ из аналогичных рассуждений в эпициклическом приближении получается, что
$ω_R = ω_x + ω_y$, то есть они попадут как раз на верхний луч семейства $\text{bl}_\text{o}$. В ходе построения
классификация я пытался уточнить процедуру, переклассифицируя такие орбиты в $\text{bl}_\text{u}$. Однако, как
было продемонстрировано на примерах орбит в главе~\ref{chap:orbits}, для семейства с $ω_R > 2 ω_x$ связь между
формой орбиты и её частотами неоднозначна. Изучение этого вопроса является предметом для будущих работ.

Заканчивая эту главу, я бы хотел ещё раз подчеркнуть, что оба найденных семейства с $ω_R \neq 2ω_x$ вносят свой вклад в
округлую морфологию. Хорошей иллюстрацией этого факта является рис.~\ref{fig:lensapart}, где из всей модели постепенно
исключаются орбиты, не поддерживающие бар: $\text{bl}_\text{u}$, $\text{bl}_\text{u}$ и $x_2$. При исключении
первого из них, внутренние изофоты теряют округлую форму, но внешние всё ещё соответствуют барлинзе.
Лишь при исключении обоих не принадлежащих классическому бару семейств и орбит $x_2$, бар приобретает арахисообразную форму.
\begin{figure}[htpb]
  \centering
  \begin{tikzpicture}[
    snap/.style = {inner sep=0pt, text width = 0.24\textwidth},
    node distance = 0.1mm
  ]
  \node[snap] (all) at (0,0) {\includepgfgraphics[width=\textwidth]{img/discuss/lensbuilding/all/xy_hex.pgf}};
  \node[snap] (no blu)   [right=of all] {\includepgfgraphics[width=\textwidth]{img/discuss/lensbuilding/no-under/xy_hex.pgf}};
  \node[snap] (no bl)    [right=of no blu] {\includepgfgraphics[width=\textwidth]{img/discuss/lensbuilding/no-under-overbar/xy_hex.pgf}};
  \node[snap] (no bl x2) [right=of no bl]  {\includepgfgraphics[width=\textwidth]{img/discuss/lensbuilding/no-under-overbar-x2/xy_hex.pgf}};
  \end{tikzpicture}
  \caption{<<Разборка>> линзы по семействам для модели BL. Приведены $(x,y)$ проекции в квадрате $[-2;2]\times [-2;2]$.
  Слева направо: все частицы в модели, без семейства $\text{bl}_\text{u}$, без $\text{bl}_\text{u}$ и $\text{bl}_\text{u}$, 
без $\text{bl}_\text{u}$, $\text{bl}_\text{u}$ и $x_2$. Видно, как влияет каждое из семейств на морфологию.}%
  \label{fig:lensapart}
\end{figure}
\end{document}
% vim:wrapmargin=3
