\documentclass[tikz]{trlnotes}
\usepackage{trmath}
\addcompatiblelayout{thesis}
\setlayout{thesis}
\usepackage{trthm}
\usepackage{trsym} 
\usepackage{trphys}
\usepackage{silence}
\WarningFilter{latex}{Reference}
\graphicspath{{../../}}
\usepackage[
backend=biber,
sorting=none,
natbib=true,
style=authoryear-comp,
language=russian
]{biblatex}
\addbibresource{../../thesisrefs.bib}
\begin{document}
\newlength{\imageheight}
\imageheight=5.5cm
В центрально-симметричных гравитационных потенциалах, отличных от закона обратных квадратов, большая часть ограниченных орбит, за редким исключением, представляет собой розетки~"--- эллипсы, апоцентры которых прецессируют вокруг начала координат \citep{book:14857}. Однако, в общем случае ниоткуда не следует, что скорости прецессии будут одинаковыми, тем не менее именно такая особенность наблюдается в галактиках с баром.
Цитируя \cite{sellwood2014a}, можно сказать, что <<\ldots оказываемый баром эффект состоит в установлении общей скорости
прецессии для орбит, которые в ином случае прецессировали бы с разной скоростью>>.
Подобное простое рассуждение помогает понять, что в моделях с баром (несмотря на всю сложность реального потенциала галактики)
должно присутствовать резонансное орбитальное семейство, существование которого обусловлено самим присутствием бара и которое поддерживает его.

В классической работе \citet{dezeeuw1985} были определены основные типа орбит в потенциалах, создаваемых вращающимися трёхосными эллипсоидами\note{В системе отсчёта связанной с его главными осями}: <<трубки>>, которые вращаются вокруг его наибольшей и наименьшей осей, и <<ящики>>, которые заметают область, включающую центр.  Многочисленные исследования аналитических моделей баров и баров в $N$-body моделях показывают, что основной тип орбит, поддерживающих бар в плоскости диска~"--- трубкообразная орбита, вращающаяся вдоль оси $z$, перпендикулярная плоскости диска и вытянутая вдоль большой оси бара \citep{athanassoula2003}. Согласно терминологии, введённой в \citet{contopoulos1980a}, принято обозначать такую
орбиту как $x_1$. Для трёхмерных баров ситуация становится более сложной, и необходимо рассматривать возмущённые в
вертикальном направлении орбиты $x_1\,v_i$, $i=1,2,3, \dotsc$,
ответвляющиеся от изначальной плоской орбиты на
$ω_z/ω_x = 2,3, \dotsc$ \citep{skokos2002a,pfenniger1991}. Однако, в данной работе нас практически не интересует
вертикальная структура B/PS балджа, поэтому более высокие, чем $x_1v_2$ резонансы здесь не рассматриваются.

Помимо $x_1$, в работе \citet{contopoulos1980a} выделяют ещё три семейства трубок в плоскости диска, вытянутых перпендикулярно бару: $x_2$, $x_3$, $x_4$ (\cite[стр.~185]{2008gady.book.....B}. Интерес для данной работы представляют только только $x_2$ и $x_4$, поскольку, как отмечают авторы оригинальной работы, слишком вытянутые орбиты $x_3$ неустойчивы. Основная разница между $x_2$ и $x_4$ заключается в знаке момента импульса вокруг оси $z$: первые
вращаются туда же, куда и бар в инерциальной системе отсчёта, а вторые~"--- ретроградно. Отдельные трубкообразные орбиты,
извлечённые из наших моделей, представлены на рис.~\ref{fig:orbtubes} вместе со спектрами временных рядов их декартовых координат. 
Если сравнить $(x,y)$ проекции орбит с изображениями выделенных орбитальных семейств (\ref{fig:xy_collage_BL}), то можно убедиться, что для $x_1$ <<петли>> на концах орбиты, находящиеся на $|x|\approx 1$ соответствуют местам уярчения для всего семейства. Такой эффект возникает из-за того, что в петлях звезда проводит больше времени.
Интересно отметить, что в модели X я не нашёл ни одной орбиты $x_2$, но при этом есть небольшое количество ретроградных орбит $x_4$. Такой же результат был получен в \citet{valluri2016,voglis2007}, использовавших модели, начальные условия которых не включали классический балдж. При этом, во всех рассмотренных моделях с балджем, наблюдается обратное соотношение, и орбит $x_2$ гораздо больше чем $x_4$.

Помимо <<трубок>>, модели включают значительное количество ящикообразных орбит, вносящих свой вклад в <<классический>> бар. На рис.~\ref{fig:orbboxy} приведён пример непериодической орбиты из модели с барлинзой (в верхнем ряду) и квазипериодические орбиты из 3-х других моделей, соответствующие наиболее высокому пику в интервале $(1.1;2)$ на одномерном распределении $ω_y/ω_x$ (\ref{fig:modelcomp1dist}).
Подобные орбиты достаточно широко освещены в литературе, например на двумерных распределениях отношений частот, изображённых
на рис.~12 из \cite{gajda2016} и рис.~13 из \cite{valluri2016} хорошо видны прямые, соответствующие тем же
значениям $ω_y/ω_x$, выступающие на фоне <<моря>> хаотических  ящикообразных орбит.

До сих пор я затрагивал только орбиты, принадлежащие к бару в его классическом понимании: с $ω_R/ω_x \approx 2$.
Однако, в наших моделях, как было описано в предыдущей главе, есть и другие семейства, не поддерживающие бар.
На рис.~\ref{fig:orbblu} приведены примеры орбит семейства $\text{bl}_{\text{u}}$ с $ω_R/ω_x < 1.9$. Большая часть 
входящих в него орбит~"--- это розетки (верхние два ряда) разных размеров, распроложенные вблизи пика в $ω_R/ω_x \approx 1.7$ на рис.~\ref{fig:modelcomp1dist}.
Сравнивая очертания подобной орбиты с изофотами семейства на \ref{fig:xy_collage_BL}, легко понять, что именно они отвечают
за характерную округлую морфологию всего семейства. Однако, не менее интересные орбиты представлены в двух нижних рядах рис.~\ref{fig:orbblu}~"--- трёх- и четырёхлистники. Подобные резонансные орбиты можно найти на рис.~22 \citet{voglis2007}, но, в целом, они незаслуженно обделены вниманием в литературе.
Возможная причина этого в том, что данное семейство становится заметным только при добавлении балджа в модель, как хорошо
заметно на \ref{fig:modelcomp1dist}, а орбитальный состав таких моделей практически никем не исследовался. Фактически это делается впервые в данной работе. 
Обращаясь к спектрам этих орбит, стоит отметить, что, в отличие от орбит лежащих в баре, у них есть выраженный второй пик на периодограмме $x(t)$. Этот факт играет определяющую роль в том, что  равенство $ω_R = 2\,ω_x$, присущее для орбит бара, не выполняется.

В отличие от семейства $\text{bl}_{\text{u}}$, огибающая орбит --- членов семейства $\text{bl}_{\text{o}}$ (рис.~\ref{fig:orbblo}) --- 
имеет, скорее, квадратную форму, что согласуется с формой изофот семейства на рис.~\ref{fig:xy_collage_BL}. Орбита, слегка похожая на орбиту во втором ряду, приводится, например, на рис.~6 в работе \cite{gajda2016}. Поскольку эти орбиты находятся
гораздо дальше от центра и <<отстают>> от бара, за весь исследуемый промежуток времени они совершают меньше 20 оборотов и их спектры зашумлены. Поэтому сложно говорить о наличии или отсутствии в спектрах вторых пиков. Однако, среди представителей этого семейства можно обнаружить и округлые орбиты (в третьем ряду рис.~\ref{fig:orbblo}) и даже резонансные трилистники (в четвёртом ряду). 
На мой взгляд именно эти орбиты отвечают за округлую форму изофот во внутренних областях семейства. 
Интересно сопоставить разные орбиты из семейства $\text{bl}_\text{o}$ с <<лучами>>, соответствующими различным орбитальным семействам на распределении \ref{fig:fRfxfRfy_bar}. 
Например, в работе \cite{gajda2016} описано семейство с квадратной морфологией с $ω_x + ω_y = ω_R$, совпадающее с верхним 
лучом на этом рисунке. Однако, приведённые примеры орбит ясно показывают отсутствие выраженной связи между внешним видом орбиты и отношением $ω_y/ω_x$.
В связи с этим, в этой работе обе группы орбит --- и c $ω_y/ω_x \approx 1$, и с $ω_y/ω_x > 1.1$ --- отнесены к одному семейству. 
Обсуждение возможных уточнений подобной классификации для того, 
чтобы всё-таки разделить отличающиеся по морфологии семейства будет приведено в следующей главе.

\begin{figure}
  \includepgfgraphics[height=\imageheight]{img/orbits/bulge1.5c/x1/orbit-2119664-xy.pgf}
  \includepgfgraphics[height=\imageheight]{img/orbits/bulge1.5c/x1/orbit-2119664-spectra.pgf}
  \\
  \includepgfgraphics[height=\imageheight]{img/orbits/bulge1.5c/x1/orbit-1889117-xy.pgf}
  \includepgfgraphics[height=\imageheight]{img/orbits/bulge1.5c/x1/orbit-1889117-spectra.pgf}
  \\
  \includepgfgraphics[height=\imageheight]{img/orbits/bulge1.5c/x2/orbit-3291205-xy.pgf}
  \includepgfgraphics[height=\imageheight]{img/orbits/bulge1.5c/x2/orbit-3291205-spectra.pgf}
  \\
  \includepgfgraphics[height=\imageheight]{img/orbits/nbulge1.5/orbit-0218273-xy.pgf}
  \includepgfgraphics[height=\imageheight]{img/orbits/nbulge1.5/orbit-0218273-spectra.pgf}
  \caption{Примеры <<трубкообразных>> орбит в исследованных моделях. В двух первых рядах представлены $x_1$-подобные орбиты, вытянутые вдоль большой оси бара из модели BL, в третьем ряду орбита типа $x_2$ из той же модели BL, а в четвёртом ряду орбита $x_4$ (ретроградная) из модели X. Слева --- проекции орбит на плоскость $x,y$. Справа --- спектры трёх частот.}
  \label{fig:orbtubes}
\end{figure}


\begin{figure}
  \includepgfgraphics[height=\imageheight]{img/orbits/bulge1.5c/box/orbit-3578359-xy.pgf}
  \includepgfgraphics[height=\imageheight]{img/orbits/bulge1.5c/box/orbit-3578359-spectra.pgf}
  \\
  \includepgfgraphics[height=\imageheight]{img/orbits/nbulge1.5/orbit-3762774-xy.pgf}
  \includepgfgraphics[height=\imageheight]{img/orbits/nbulge1.5/orbit-3762774-spectra.pgf}
%   \\
%   \includepgfgraphics[height=\imageheight]{img/orbits/bulge1.5/orbit-0917708-xy.pgf}
%   \includepgfgraphics[height=\imageheight]{img/orbits/bulge1.5/orbit-0917708-spectra.pgf}
  \\
  \includepgfgraphics[height=\imageheight]{img/orbits/bulge1.5/orbit-3521736-xy.pgf}
  \includepgfgraphics[height=\imageheight]{img/orbits/bulge1.5/orbit-3521736-spectra.pgf}
  \\
  \includepgfgraphics[height=\imageheight]{img/orbits/bulge1.5c/boxlet/orbit-1467792-xy.pgf}
  \includepgfgraphics[height=\imageheight]{img/orbits/bulge1.5c/boxlet/orbit-1467792-spectra.pgf}
\caption{<<Ящикообразные>> орбиты из рассматриваемых моделей. В первом ряду непериодическая орбита из модели BL, во втором ряду квазирезонансная орбита с $ω_y/ω_x = 5/3$ из модели X, в третьем квазирезонансная орбита с $ω_y/ω_x = 3/2$ из модели BLx, в четвёртом квазирезонансная орбита с $ω_y/ω_x = 4/3$ из модели BL. Слева --- проекции орбит на плоскость $x,y$. Справа --- спектры трёх частот.}
\label{fig:orbboxy}
\end{figure}

\begin{figure}
  \centering
  \includepgfgraphics[height=\imageheight]{img/orbits/bulge1.5c/rosetta/orbit-0218811-xy.pgf}
  \includepgfgraphics[height=\imageheight]{img/orbits/bulge1.5c/rosetta/orbit-0218811-spectra.pgf}
  \\
  \includepgfgraphics[height=\imageheight]{img/orbits/bulge1.5c/rosetta/orbit-3793601-xy.pgf}
  \includepgfgraphics[height=\imageheight]{img/orbits/bulge1.5c/rosetta/orbit-3793601-spectra.pgf}
  \\
  \includepgfgraphics[height=\imageheight]{img/orbits/bulge1.5c/nfoil/orbit-2598721-xy.pgf}
  \includepgfgraphics[height=\imageheight]{img/orbits/bulge1.5c/nfoil/orbit-2598721-spectra.pgf}
  \\
  \includepgfgraphics[height=\imageheight]{img/orbits/bulge1.5c/nfoil/orbit-3533413-xy.pgf}
  \includepgfgraphics[height=\imageheight]{img/orbits/bulge1.5c/nfoil/orbit-3533413-spectra.pgf}
  \caption{Примеры орбит семейства $\text{bl}_\text{u}$ в модели BL. Слева --- проекции орбит на плоскость $x,y$. Справа --- спектры трёх частот. Виден заметный второй пик в спектре $x(t)$}
  \label{fig:orbblu}
\end{figure}


\begin{figure}
  \includepgfgraphics[height=\imageheight]{img/orbits/bulge1.5c/sq/orbit-0609211-xy.pgf}
  \includepgfgraphics[height=\imageheight]{img/orbits/bulge1.5c/sq/orbit-0609211-spectra.pgf}
  \\
  \includepgfgraphics[height=\imageheight]{img/orbits/bulge1.5c/sq/orbit-3404886-xy.pgf}
  \includepgfgraphics[height=\imageheight]{img/orbits/bulge1.5c/sq/orbit-3404886-spectra.pgf}
  \\
  \includepgfgraphics[height=\imageheight]{img/orbits/bulge1.5c/cr/orbit-2403647-xy.pgf}
  \includepgfgraphics[height=\imageheight]{img/orbits/bulge1.5c/cr/orbit-2403647-spectra.pgf}
  \\
  \includepgfgraphics[height=\imageheight]{img/orbits/bulge1.5c/cr/orbit-2193133-xy.pgf}
  \includepgfgraphics[height=\imageheight]{img/orbits/bulge1.5c/cr/orbit-2193133-spectra.pgf}
  \caption{Примеры орбит семейства $\text{bl}_\text{o}$ в модели
  BL. Слева --- проекции орбит на плоскость $x,y$. Справа ---
спектры трёх частот. Для некоторых орбит заметны вторые пики, но
для большинства спектров их нельзя достоверно отождествить.}
  \label{fig:orbblo}
\end{figure}

\end{document}
% vim:wrapmargin=3
