\documentclass[tikz]{trlnotes}
\usepackage{trmath}
\addcompatiblelayout{thesis}
\setlayout{thesis}
\usepackage{trthm}
\usepackage{trsym} 
\usepackage{trphys}
\usepackage{silence}
% \usepackage{tikz}
\usepackage[
backend=biber,
sorting=none,
natbib=true,
style=authoryear-comp,
language=russian
]{biblatex}
\addbibresource{../../thesisrefs.bib}
\WarningFilter{latex}{Reference}
\graphicspath{{../../}}
\begin{document}
В работах, посвящённых классификации орбит на основе доминирующих частот, часто используются набор из угловой
частоты $Ω$, эпициклической частоты $ϰ$ и вертикальной частоты $ν$
\citep{athanassoula2002a,ceverino2007,voglis2007}. Однако для изучения структуры бара во вращающейся системе
отсчёта гораздо удобнее рассматривать <<декартовы>> ($ω_x, ω_y, ω_z, ω_R$) частоты, введённые в предыдущей главе, и не использовать угловую частоту $Ω$, определение которой не очень точно и требует учёта дополнительной информации кроме спектров
\citep{athanassoula2002,gajda2016}. К тому же, \citet{valluri2016} показали, что, по крайней мере, в <<замороженных>> потенциалах отношения декартовых частот
гораздо лучше разделяют орбитальные семейства. 
Однако, все-таки полезно сопоставить одни частоты с другими для лучшего понимания физики происходящего. Соотношения между ними такие: 
\[
  ω_x \approx Ω - Ω_p, \quad ω_z = ν, \quad ω_R = ϰ,
\]
где $Ω_p$~"--- скорость вращения бара. Из-за постепенного замедления вращения бара и возможного вклада вторых частот первое равенство оказывается приближённым, а остальные два верны просто по определению.

Перед тем как начать классифицировать семейства, необходимо сузить рассматриваемое множество частиц до наиболее
интересных. Предварительная обработка модели включала следующие этапы.
\begin{enumerate}
  \item Отделение диска от гало. 
  \item Перенос начала отсчёта координат в центр масс диска.
  \item Переход в систему отсчёта, вращающуюся вместе с баром.
  \item Выбор интересующей нас области диска, где расположен бар.
\end{enumerate}
Первые три этапа были выполнены с помощью пакета \texttt{NEMO}, после чего был произведено вычисление доминирующих частот, как об этом было сказано в главе \ref{chap:method}. Опишем теперь более 
подробно четвёртый шаг.
\paragraph{Отделение диска и выбор области, связанной с баром}

В силу выбора координатных осей, связанных с баром, в модели возникает характерный радиус, на котором угловая скорость вращения системы отсчёта практически совпадает с угловой скоростью вращения частиц в диске.
В окрестности этого радиуса абсолютные значения доминирующих
орбитальных частот $\omega_x$ и $\omega_y$ окажутся малыми, порядка $\frac{2\pi}{t_2 - t_1}$, поскольку частица не успевает совершить полный оборот вокруг центра на отрезке времени, для которого строится временной ряд и вычисляется преобразование Фурье. Естественно отождествить этот радиус с радиусом
коротации в модели (CR). Обозначим его через $R_c$ и выделим три группы частиц по их расположению относительно коротации
(Рис.~\ref{fig:disksplit}, внизу)\note{В качестве вертикальной оси
выбран $|ω_x|$, так как именно эта величина вычисляется
алгоритмом. В дальнейшем изложении модуль опускается, поскольку
внутри коротации $ω_x > 0$.}.
\begin{enumerate}
  \item $|ω_x| < |ω_{x, \min}| \:\lor\: |ω_y| < |ω_{x, \min}|$.
Частицы из этой группы образуют кольцо с утолщениями в
симметричных точках на оси $y$ вблизи радиуса коротации (оно закрашено в верхней части рис.~\ref{fig:disksplit} оранжевым цветом).
Очень похожая структура была обнаружена в работе \citet{ceverino2007}, но по критерию $Ω \approx Ω_p$, который, на мой взгляд, сложнее. В упомянутой работе подробно описывается <<захват>> частиц на коротации, который
наблюдается и в наших моделях. С точки зрения аналитических моделей потенциалов, он связан с устойчивостью
точек Лагранжа $L_4$ и $L_5$ \citep[Fig.~3.14]{2008gady.book.....B}; утолщения на кольце наблюдаются в окрестности этих точек.
В качестве радиуса коротации принималась середина кольца.  
\item Частицы с $|ω_x| > |ω_{x, \min}| \land |ω_y| > |ω_{y, \min}|$, находящие дальше коротации, связаны с диском
(рис.~\ref{fig:disksplit}, область, закрашенная зелёным цветом). Поскольку нас интересует только орбитальный состав бара, они были сразу исключены из рассмотрения. 
\item Частицы с $|ω_x| > |ω_{x, \min}| \land |ω_y| > |ω_{y, \min}|$,
  находящие внутри коротации (рис.~\ref{fig:disksplit}, область, закрашенная голубым цветом), представляют наибольший интерес в данном
  исследовании. В классической работе \citet{contopoulos1980} было
  показано, что основные орбиты, составляющие
бар не могут заходить за радиус коротации, поэтому, ограничив рассмотрение только этой областью, мы ничего не
упустим.
\end{enumerate}

\begin{figure}[htpb]
  \centering
  \includepgfgraphics[width=0.77\linewidth]{img/families/bulge1.5c/discsplit-proj.pgf}\par
  \includepgfgraphics[width=0.77\linewidth]{img/families/bulge1.5c/discsplit-cr.pgf}
  \caption{Разбиение частиц на 3 области области на примере модели BL.
    Верхний рисунок иллюстрирует разбиение в проекции ($x$, $y$).
    На нижнем --- показан профиль величины $|ω_x| \approx |Ω - Ω_p|$ (в зависимости от среднего радиуса орбиты) и схематично продемонстрирован выбор радиуса коротации. Все рисунки относятся к моменту времени $t=450$.
Детали процедуры разбиения описаны в тексте.}%
  \label{fig:disksplit}
\end{figure}

Однако, как можно заметить на рис.~\ref{fig:disksplit}, даже во внутренней части всё ещё остаются частицы, мало
связанные с баром. Особенно хорошо это заметно в проекции $(x, z)$, где, помимо утолщённой части бара, остаются
частицы лежащие вблизи $z=0$. Нельзя исключать возможность, что такие частицы все же принадлежат бару (это могуть быть т.н.
ansae~"--- <<ручки>>, соединяющие бар с кольцом), поскольку размеры B/PS структуры меньше длины самого бара. Однако, как
уже было описано во Введении, барлинзы и B/PS балджи имеют сходные размеры, поэтому возможная потеря концов бара
вряд ли изменит его морфологию, наблюдаемую <<плашмя>>. 

\begin{figure}[htpb]
  \centering
  \begin{tikzpicture}[
    im/.style = {inner sep=0pt, text width=0.8\textwidth},
    node distance = 10pt
    ]
    \node[im] (FxFz) at (0,0) [anchor=south west] {\includepgfgraphics[width=\textwidth]{img/families/bulge1.5c/cl_fxfz.pgf}}; 
    \node[im] [below=of FxFz] {\includepgfgraphics[width=\textwidth]{img/families/bulge1.5c/m_Ravfzfx.pgf}}; 
    \begin{scope}[x={(FxFz.south east)},y={(FxFz.north west)}]
      \coordinate (origin) at (0.126,0.111);
      \coordinate (x1y2) at (0.87,0.97);
      \draw[very thick,red!30!black] let \p1 = ($ (x1y2) - (origin) $) in 
        (origin)  -- + (0.89*\x1, \y1);
    \end{scope}
  \end{tikzpicture}
  \caption{Двумерные карты частот. Вверху: двумерное распределение
    $ω_x$ -- $ω_z$. Тёмно-красная прямая на рисунке соответствует условию
    $ω_z = 2.25\, ω_x$ (исключается все слева от неё). Внизу: двумерное
    распределение отношения $ω_z/ω_x$ -- средний радиус
  орбиты $\averg{R}$ (исключается область выше $ω_z/ω_x = 2.25$ и правее $\averg{R} = 0.8$).}
  \label{fig:innerrefine}
\end{figure}

Опишем процедуру выбора исследуемой внутренней области диска. На верхней части рис.~\ref{fig:innerrefine} можно
заметить выделяющуюся прямую, соответствующую орбитам с $ω_z = 2\, ω_x$, <<полосу>> правее неё с меньшим отношением
$ω_z /ω_x$ и область вблизи $ω_x = 0$ с большими значениями этого отношения. Нижняя часть рис.~\ref{fig:innerrefine} ясно демонстрирует, что эта область состоит из двух частей~"--- орбиты с небольшими значениями $\averg{R}$ вблизи начала координат и <<хвост>> орбит, уходящих далеко от центра. Именно эти орбиты лежат вблизи $z=0$, как хорошо видно на рис.~\ref{fig:inneredgeon} (картинки справа). В итоге, из изначально выбранной области
внутри коротации дополнительно отбираются орбиты по критериям
\begin{equation}
  ω_z/ω_x < 2.25 \: \lor \: \averg{R} < 0.8.
\end{equation}
В дальнейшем я буду изучать только ту часть области внутри коротации, для
которой выполнен вышеприведённый критерий, поскольку остальная часть
диска не представляет интереса для исследования орбитального состава
внутренних областей бара, наблюдаемого <<плашмя>>. При этом, для всех
четырёх моделей рассматриваемая часть содержит около $50\%$ от 4 млн.
частиц диска. Для неё было принято обозначение ROI. 
\begin{figure}[htpb]
  \centering
  \includepgfgraphics[width=0.48\linewidth]{img/families/bulge1.5c/xy_hex.pgf}
  \includepgfgraphics[width=0.48\linewidth]{img/families/bulge1.5c/other/xy_hex.pgf}
  \\
  \includepgfgraphics[width=0.48\linewidth]{img/families/bulge1.5c/xz_hex.pgf}
  \includepgfgraphics[width=0.48\linewidth]{img/families/bulge1.5c/other/xz_hex.pgf}
  \caption{Проекции частиц, расположенных внутри коротации, на плоскости $(x,y)$ (вверху) и $(x,z)$ (внизу) на момент времени $t=450$. Слева: отобранная для дальнейшего анализа область, справа: оставшаяся часть внутри коротации. Цветовая шкала совпадает на изображениях в одном ряду: для верхнего ряда максимальное значение составляет $5000$, для нижнего~"--- $7500$.%
  }%
  \label{fig:inneredgeon}
\end{figure}

\paragraph{Основные семейства}
\tikzset{
  modelname/.style = {
    fill=lightgray, fill opacity=0.8, text opacity=1,
    font=\Large
}}
Опишем в деталях процедуру выбора орбитальных семейств. 
Поскольку задачей является получение единообразной и максимально полной классификации орбит, объясняющей все рассмотренные варианты морфологии бара, необходимо сначала хотя бы на качественном уровне понять, какие семейства можно выделить во всех моделях по отношению доминирующих частот. На рис.~\ref{fig:modelcomp1dist} приведены проекции области,
выбранной по алгоритму, описанному выше, на плоскость $(x,y)$ для всех 4-х моделей (слева) и одномерные распределения отношений $ω_R/ω_x$, $ω_y/ω_x$. Хорошо заметен пик вблизи
2 для первого из распределений частот и вблизи 1 для второго. Поскольку в выбранной системе отсчёта $ω_x \approx Ω - Ω_p$, можно, следуя работам \citet{gajda2016,portail2015}, отождествить $ω_R \approx 2\, ω_x$ с баром в его классическом
понимании, составленном из орбит\note{Более подробное обсуждение типов орбит изложено в главе~\ref{chap:orbits}},
порождённых так называемым семейством $x_1$ и находящихся во внутреннем резонансе Линдблада (ILR) \citep{athanassoula2003}.
Важно отметить, что использование критерия, завязанного на декартовы частоты, не искажает морфологию получившегося <<классического>> бара. Представленные в дальнейшим изложении
проекции этого семейства очень похожи на изображения ILR в работе \citet[Fig.~10]{ceverino2007}, отождествлённого по условию $Ω - Ω_p \approx ϰ/2$. Слева и справа от пика в $2$ на распределении $ω_R/ω_x$ отчётливо видны ещё как минимум два семейства:
с $ω_R/ω_x > 2$ и $ω_R/ω_x < 2$.  По мере перехода от арахисообразной формы бара к бару с линзой их
относительный вклад увеличивается, а количество частиц, находящихся на
ILR уменьшается. 
% нет, не было показано.
%Наблюдается важное отличие между моделями BL и BLx~"---
%для семейства с $ω_R/ω_x < 2$ возникает довольно широкий пик на распределении $ω_R/ω_x$ вблизи $ω_R/ω_x \approx 1.7$. Как будет показано в дальнейшем, именно семейство, сооотвествующее этому пику, вносит наибольний вклад в округлую форму, присущую барлинзе.

Не менее интересная закономерность наблюдается и на распределениях отношения частот $ω_y/ω_x$. Широкая область справа от пика в единице,
соответствующая т.н. boxy орбитам (см., например, \citealp{valluri2016}), становится всё менее протяжённой по значению этого отношения и менее населённой. На каждом графике в этой области
можно выделить свой небольшой пик, соответствующий резонансным орбитам. В терминологии, принятой в статье \cite{valluri2016}, такие орбиты
называются resonant boxlet орбитами. Для модели с арахисообразным баром маленький пик соответствует $ω_y/ω_x=5/3$, для промежуточных морфологий~"--- $3/2$, а для бара с барлинзой~"--- $4/3$.
За счёт уменьшения количества boxy орбит возрастает величина пика с $ω_y/ω_x = 1$. Значительную его часть
составляют орбиты, генетически связанные с семейством $x_1$ и ответвлениями от него в вертикальном направлении, находящиеся в баре.

Обобщая качественные рассуждения, изложенные выше, можно,  сформулировать следующие критерии <<автоматической>> (в
таком же смысле, как в \cite{valluri2016}) классификации орбитальных семейств. Эти критерии вместе с условным обозначением для каждого семейства приведены ниже.
Границы семейств были выбраны на основе результатов статьи \citet{portail2015}, поскольку другие методы (например, по погрешности отношения) настолько же нефизичны и не позволяют получить универсальную оценку, не зависящую от радиуса.
\begin{table}[htpb]
    \centering
  \begin{tabular}{r|l}
    \toprule
    {семейство}   & критерий \\
    \midrule
    ILR & $|ω_R/ω_x - 2| < 0.1$ \\
    $x_i$ bar & $|ω_R/ω_x - 2| < 0.1 \land |ω_y/ω_x - 1| < 0.1$ \\
    boxy bar  & $|ω_R/ω_x - 2| < 0.1 \land  ω_y/ω_x > 1.1$ \\
    $\text{bl}_\text{u}$  & $ω_R/ω_x < 1.9$ \\
    $\text{bl}_\text{o}$  & $ω_R/ω_x > 2.1$ \\
    \bottomrule
  \end{tabular}
  \caption{Обозначения и критерии выделения для рассматриваемых орбитальных семейств.}
  \label{tab:orbfamiles}
\end{table}

\begin{figure}[htpb]
\centering
\begin{tikzpicture}[
  snap/.style = {inner sep=0pt, text width = 0.2\textwidth},
  dist/.style = {inner sep=0pt, text width = 0.4\textwidth},
  node distance = 0.1cm,
  ]
  \node[snap] (X)     at (0,0)     {\includepgfgraphics[width=\textwidth]{img/families/nbulge1.5/xy_hex_nl.pgf}};
  \node[snap] (Xbl) [below=of X,  yshift=-1.5cm]{\includepgfgraphics[width=\textwidth]{img/families/bulge1.5m/xy_hex_nl.pgf}};
  \node[snap] (BLx) [below=of Xbl,yshift=-1.5cm]{\includepgfgraphics[width=\textwidth]{img/families/bulge1.5/xy_hex_nl.pgf}};
  \node[snap] (BL)  [below=of BLx,yshift=-1.5cm]{\includepgfgraphics[width=\textwidth]{img/families/bulge1.5c/xy_hex_nl.pgf}};
  \foreach \modelname\alias in {X/nbulge1.5, Xbl/bulge1.5m, BLx/bulge1.5, BL/bulge1.5c}{
    \node[modelname, anchor=north west, xshift=2pt, yshift=-2pt] at (\modelname.north west) {\modelname}; 
    \node[dist] (\modelname fy/fx) [right=of \modelname,yshift=-0.2cm] {\includepgfgraphics[width=\textwidth]{img/families/\alias/p_dNfyfx.pgf}};
    \node[dist] (\modelname fR/fx) [right=of \modelname fy/fx]         {\includepgfgraphics[width=\textwidth]{img/families/\alias/p_dNfRfx.pgf}};
  }
\end{tikzpicture}  
\caption{Морфология бара и одномерные распределения отношений доминирующих частот для всех 4 моделей. В левой колонке:
  $(x,y)$ проекции интересующей области в квадрате $[-3;3]\times[-3;3]$, в центре: распределение отношения $ω_y/ω_x$ с шириной бина $0.01$, справа: распределение отношения $ω_R/ω_x$ с шириной бина $0.01$. По мере перехода от арахисообразного бара (X) к барлинзе (BL) заметно понижение высоты пика, соответствующего бару в классическом
  смысле, и увеличение высоты пиков для не принадлежащих ему семейств. Уменьшается также и количество boxy орбит.}
\label{fig:modelcomp1dist}
\end{figure}

Более наглядно разбиение на семейства видно на двумерных картах отношений
доминирующих частот, показанных на рис.~\ref{fig:fRfxfRfy_bar}.
Сопоставляя выделенные семейства с <<лучами>> на иллюстрации в его правой
части, можно заметить, что орбиты, попавшие в $\text{bl}_{\text{o}}$,
разбиваются на две группы: с $ω_x + ω_y \approx ω_R$ и с $ω_x \approx
ω_y$. Семейство из первой группы было отмечено в работе \citet{gajda2016}
как семейство, не поддерживающее бар. Но при этом авторы включают в него
и не попавшие ни в одну из выделенных ими групп орбиты с $ω_y/ω_x < 1.2$,
для которых амплитуда наибольшего пика на периодограмме $y$ координаты
больше аналогичной величины для $x$. Внимательно изучив форму изофот на
проекциях всех частиц семейств, я не обнаружил реальной разницы между
верхним и нижним лучом. По этой причине в данной работе обе этих группы
объединены в одно семейство  $\text{bl}_\text{o}$. Примеры орбит,
приведённые в следующей главе, подтверждают осмысленность такого решения.

\begin{figure}
\centering
% \begin{minipage}{.23\textwidth}
\begin{tikzpicture}[
  annotation/.style = {
    fill=lightgray, fill opacity=0.8, text opacity=1,
    rounded corners=3pt
  },
  preview/.style = {inner sep=0pt, text width=0.2017\textwidth},
  overview/.style = {inner sep=0pt, text width=0.78\textwidth},
  node distance = 0pt
  ]
  \node[overview, anchor=south west] (BL) at (0,0) %
    {\includepgfgraphics[width=\textwidth]{img/families/bulge1.5c/2dpr_fxfRfyfR.pgf}};
  
  \node[preview]  (Xb)  [left=of BL,xshift=-0.018\textwidth, yshift=0.029\textwidth] %
    {\includepgfgraphics[width=\textwidth]{img/families/bulge1.5m/2dpr_fxfRfyfR_nl.pgf}};
  \node[preview]  (X)   [above=of Xb]  %
    {\includepgfgraphics[width=\textwidth]{img/families/nbulge1.5/2dpr_fxfRfyfR_nl.pgf}};
  \node[preview]  (BLX) [below=of Xb] {\includepgfgraphics[width=\textwidth]{img/families/bulge1.5/2dpr_fxfRfyfR_nl.pgf}};
  % family annotations for barlens
  \begin{scope}[x={(BL.south east)},y={(BL.north west)}]
    \coordinate (cen) at (0.427, 0.48);
    \node (x1bar) [annotation,anchor=west]       at (0.53, 0.50) {$x_i$ bar};
    \node         [annotation,anchor=center]     at (0.43, 0.70) {boxy bar};
    \node         [annotation,anchor=west]       at (0.65, 0.70) {$\text{bl}_\text{u}$};
    \node         [annotation,anchor=north west] at (0.30, 0.50) {$\text{bl}_\text{o}$};
    \draw[->,thick,red!30!black,shorten >=2pt] (x1bar.west) to[out=180, in=320] (cen);
    \node         [modelname,anchor=north west] at (0.105, 0.98) {BL};
  \end{scope}
  \foreach \modelname in {Xb, X, BLX}{
    \node [modelname, anchor=north west, xshift=2pt, yshift=-2pt] at (\modelname.north west) {\modelname};
  }
\end{tikzpicture}
\caption{Двумерное распределение орбит на
плоскости $\omega_{\mathrm{R}}/ω_x$--$ω_y/ω_x$ для всех 4 моделей. Для
модели с барлинзой (BL) подробно отмечены выделяемые семейства. Для
остальных моделей распределения оформлены врезкой с такой же сеткой и
цветовой шкалой. Видно, что все семейства, выделенные для BL, присутствуют и в остальных моделях, но их населённость другая.}
\label{fig:fRfxfRfy_bar}
\end{figure}

Однако, внутри основного семейства, поддерживающего бар, морфологически выделяется два подсемейства,
прародителями которых могут быть резонансные орбиты $x_1$ и $x_2$, соответственно. Для того, чтобы учесть
эти различия, необходимо привлечь дополнительную информацию, кроме частот.
В качестве такой дополнительной информации я использовал двумерные распределения отношения средней протяжённости
орбиты по осям $y$ и $x$ к её среднему радиусу, представленные на рис.~\ref{fig:x1x2sep}. На этом рисунке хорошо видны две отдельные области, соответствующие разным типам орбит. Подобная картина наблюдается и для семейств
$\text{bl}_\text{u}$ и $\text{bl}_\text{o}$\note{В последнем случае между <<островами>> есть небольшой
<<перешеек>>, посередине которого и проходит <<демаркационная>> линия.}, однако для них вытянутые семейства содержат относительно малую долю частиц и не поддерживают ни бар, ни линзообразную структуру и, поэтому, были просто убраны из анализа. 
Поскольку для каждого семейства (и каждой модели) граница, разделяющая области, своя, я не буду приводить точные значения коэффициентов уравнений прямых в
уточнённых критериях для экономии места и времени читателя.  

\begin{figure}[htpb]
  \centering
  \includepgfgraphics[width=0.7\textwidth]{img/families/bulge1.5c/bar/circ/m_Ravyavxav.pgf}
\caption{Двумерное распределение величины $\averg{|y|}/\averg{|x|}$ в зависимости от $\averg{R}$ для модели с барлинзой, где $\averg{|y|}$, $\averg{|x|}$ и  $\averg{R}$ --- средние значения модуля $x$-, $y$-координаты орбиты и её радиальной координаты. Чётко выделяются
два семейства: компактные $x_2$-подобные орбиты в верхнем левом углу и вытянутые $x_1$-подобные внизу.}%
\label{fig:x1x2sep}
\end{figure}

Точные значение количества частиц в каждом из выделенных семейств для всех моделей представлены в таблице~\ref{tab:familiesnumbers},
а общая схема итоговой автоматической классификации проиллюстрирована для модели с барлинзой на рис.~\ref{fig:xy_collage_BL}.

\begin{table}[hb]
  \centering
  \begin{tabular}{cr|ccccc}
    \toprule
              & {семейство}   & {X} & {Xbl} & {BLx} & {BL} & \\
    \midrule
    \midrule
              & ROI                  & 60.08 & 45.90 & 52.86 & 49.76 & \\
    \midrule
              & $\text{bl}_\text{o}$ & 5.36  & 7.82  & 9.98  & 13.49 & \\
              & $\text{bl}_\text{u}$ & 0.96  & 1.05  & 3.42  & 9.22  & \\
              & ILR                  & 50.40 & 32.98 & 34.16 & 22.37 & \\
    \midrule
              & boxy bar             & 45.19 & 25.68 & 23.88 & 10.45 & \\
              & $x_i$ bar            & 3.79  & 6.52  & 9.51  & 11.35 & \\
    \midrule
              & $x_1$                & 3.48  & 5.70  & 7.60  & 9.64  & \\
              & $x_2$                & 0.30  & 0.83  & 1.91  & 1.71  & \\
    \bottomrule
  \end{tabular}
  \caption{Количество орбит каждого семейства в моделях.
    ROI~"--- вся рассматриваемая область, содержащая бар, а остальные семейства определены в
    таблице~\ref{tab:orbfamiles}.
    Приведена доля от полного числа частиц в диске ($4\cdot 10^6$) в процентах.
    Под чертой указаны подсемейства, на которые разбивается семейство над чертой.
    Незначительное несовпадение сумм возникает из-за применения вышеописанной <<очистки>> к семействам
    $\text{bl}_\text{u}$ и $\text{bl}_\text{o}$ и небольшого количества нерегулярных орбит с $ω_y < ω_x$ в баре.
    Хорошо видно, как с переходом от арахисообразного бар к барлинзе увеличивается уменьшается вклад boxy bar, за
    счёт чего растёт семейство $x_i$ bar с $ω_y/ω_x \approx 1$. Растёт и доля семейств, не поддерживающих бар:
    $\text{bl}_\text{u}$, $\text{bl}_\text{o}$, причём вклад первого из них становится значительным
    только для модели $BL$.}
  \label{tab:familiesnumbers}
\end{table}

\begin{figure}
  \centering
  \begin{tikzpicture}[
      snap/.style = {text width = 5.5cm, inner sep=0pt},
      snaplabel/.style = {
        fill=white, fill opacity=0.8, text opacity=1,
        font=\Large
      },
      smaller/.style = {text width = 2.5cm, inner sep=0pt},
      node distance = 0.5cm,
      connection/.style = {fill = lightgray, draw = lightgray, thick, rounded corners=1pt}
    ]
    \newcommand*{\nodeimage}[1]{\includepgfgraphics[width=\textwidth]{img/families/bulge1.5c/#1}}
    \node[snap] (fullbar)                                      {\nodeimage{cl_xy_hex_nl.pgf}};
    \node[snap] (bar)              [below=of fullbar]          {\nodeimage{bar/xy_hex_nl.pgf}};
    \node[snap] (underbar)         [right=of fullbar]          {\nodeimage{underbar/pure/xy_hex_nl.pgf}};
    \node[snap] (overbar)          [below=of underbar]         {\nodeimage{overbar/pure/xy_hex_nl.pgf}};
    \node[snap] (bar circ)         [below=of bar]              {\nodeimage{bar/circ/xy_hex_nl.pgf}};
    \node[snap] (bar boxy)         [right=of bar circ]         {\nodeimage{bar/boxy/xy_hex_nl.pgf}};
    \node[snap] (bar circ along_x) [below=of bar circ]         {\nodeimage{bar/circ/along_x/xy_hex_nl.pgf}};
    \node[snap] (bar circ along_y) [right=of bar circ along_x] {\nodeimage{bar/circ/along_y/xy_hex_nl.pgf}};
    \foreach \snap/\label in {
      fullbar/ROI, 
      bar/ILR,
      underbar/$\text{bl}_\text{u}$,
      overbar/$\text{bl}_\text{o}$,
      bar circ/$x_i$ bar,
      bar boxy/boxy bar,
      bar circ along_x/$x_1$,
      bar circ along_y/$x_2$%
      }{
        \node [snaplabel, anchor=north west, xshift=2pt, yshift=-2.5pt] at (\snap.north west) {\label};
    }
%     \draw (fullbar.south west) -- (underbar.north west);
    \path[connection] (fullbar.south east) to[out=90,in=240]  (underbar.west)  
                                           to[out=120,in=270] (fullbar.north east) -- cycle;
    %{}
    \path[connection] (underbar.west) -- +(-6pt, +22pt) -- +(-4pt, 0pt) -- +(-6pt, -22pt) -- cycle; 
    \foreach \from/\leftto\rightto in {
      fullbar/bar/overbar,%
      bar/bar circ/bar boxy,%
      bar circ/bar circ along_x/bar circ along_y%
    }{\path[connection] (\from.east)       to[out=270,in=110] (\rightto.north west) 
                                           to[out=160,in=0]   (\from.south) -- (\from.south east) -- cycle;
      \path[connection] (\from.south west) to[out=0,in=150]   (\leftto.north) 
                                           to[out=30,in=180]  (\from.south east) -- cycle;
      \path[connection] (\rightto.north west) -- +(0, 11pt) -- +(-3pt, 3pt) -- +(-11pt, 0) -- cycle;
      \path[connection] (\leftto.north) -- +(-22pt, 6pt) -- +(0pt, 4pt) -- +(+22pt, 6pt) -- cycle;
    }
  %
  \end{tikzpicture}
  \caption{Схема разбиения на семейства на примере модели BL. Приведены $(x,y)$ проекции в квадрате $[-2;2]\times[-2;2]$ на момент времени $t=450$.}
\label{fig:xy_collage_BL}
\end{figure}


\end{document}
% vim:wrapmargin=3
